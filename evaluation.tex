\section{Evaluation}
\label{SEC:evaluation}

In order to evaluate the effectiveness of our lesson we developed a lecture
that contained both the card tricks and discussion of the topics they were
intended to teach.  As part of this effort we created a short assessment
designed to measure the improvement in participants' mastery of the material.
This assessment, shown in figure~\ref{fig:assessment}
consisted of 12 multiple choice questions (4 for each topic),
3 likhert-scale survey statements,
and a free response comments section.
We planned to have participants complete the 12 multiple choice questions twice.
Once before our lecture to gather a baseline of
participants' existing knowledge of the material and again at the conclusion of our
lecture as a summative assessment to determine if the students' mastery
improved.  Both assessments used the same questions.
Participants were asked to complete our survey statements and
free response comment section after the second assessment.  The purpose of
these statements and comments were to determine whether or not the
participants found our lecture enjoyable and engaging.

We presented our lecture as part of a high-school level computer science
program.  Our approximately 90-minute session~\footnote{conducted remotely
via Zoom due to COVID-19 restrictions} was attended by AAA students.  Of
these BBB completed our formative and summative assessments and CCC
completed both of these components in addition to our survey statements.
Our session consisted of three phases, one for each topic, each consisting
of a short introduction to the topic,
followed by a performance of the topics magic trick
and ending with a detailed discussion of the topic
using the magic trick as a scaffold.





\begin{figure}[H]
  \begin{tabular}{c | c | c | c}
    & Question Text  &     & Question Text\\
    \hline
    Q1 & Question 1  & Q2  & Question 2   \\
    Q3 & Question 3  & Q4  & Question 4   \\
    Q3 & Question 5  & Q6  & Question 6   \\
    Q1 & Question 7  & Q8  & Question 8   \\
    Q1 & Question 9  & Q10 & Question 10  \\
    Q1 & Question 11 & Q12 & Question 12  \\
    \hline
    S1 & Statement 1 & S1  & Statement 2  \\
    S1 & Statement 3 &     &              \\
  \end{tabular}
  \caption{Study Instrument}
  \label{fig:assessment}
\end{figure}

\begin{figure}[H]
  \caption{ }
  \
\end{figure}


%The PAS helped me understand where the memory for local, global, and static
%variables is allocated.
%
%The PAS helped me to see how a buffer overflow can change the values of
%variables adjacent to the buffer that was overflowed
%
%I understand process execution better after using the PAS
%
%I was able to identify concepts I did not understand after using the
%software
%
%The PAS window enhanced the course of the area of program execution
%
%The text of the PAS window was easy to read
%
%It was easy to identify one stack frame from another
%
%It was easy to identify the contents of each stack frame
%
%It was easy to identify the address and value of variables in the data
%section
%
%How long did it take to understand the process address spaces and variables
%using the software
%
%
%variables variables times did you use the software
%
%how long did you use the software in total
