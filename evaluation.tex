\section{Study Instrument and Evaluation}
\label{SEC:evaluation}

\paragraph{Method}The goal of our study was to judge how effective a non-traditional approach
could be in teaching novices about our selected attacks.  To do so,
we put together a 90 minute Zoom session to be presented at a summer
computer science workshop for high school students.  The session was
offered twice during the three week workshop, with the lead investigator
serving as the presenter and magician.  It was divided into three phases,
one
for each topic,
and each phase consisted of a performance of the topic's magic
trick, followed by a detailed discussion that used the magic trick
as a scaffold.  A total of 15 students attended the two sessions.

To measure improvements in participants' mastery of the subject, we
designed an assessment instrument (shown in Table~\ref{fig:assessment})
consisting of
12 multiple choice questions (4 for each topic),
3 Likert-scale survey statements,
and a free response comments section.
All of these materials were reviewed and approved by our university's
Institutional Review Board and, though attendance was open to all, we only
used data from students who had given us their consent to be part of the
study.

Participants completed the 12 multiple choice questions twice,
once before our lecture to gather a baseline of
participants' existing knowledge of the material,
and again at the conclusion of our
lecture as a post-test to determine if student mastery improved.
Both assessments used the same questions.
After taking part in the workshop, participants also evaluated the session
itself by offering opinions on whether or not they found our lecture
enjoyable and engaging.
The post-test was handed out at the conclusion of the lesson
and students were asked
to complete it in their own time.
Of the 15 students that attended our session, 10 agreed to participate in
the study and completed our pre-test. Of these 10, 5
students completed our post-test and survey.  Given the overwhelmingly
positive feedback we received from our session attendees, we believe the
decreased number of participants completing the post test can likely
be attributed
to the awkwardness of an interactive session online
rather than deficiencies in
the session itself.


\begin{table}[h]
  \scriptsize
  \begin{tabular}{c | p{6cm}}
Question Text  &     \\
\hline
Q1 & Which of the following is the best definition of social engineering \\
Q2  & The act of creating a scenario in order to extract information is called:   \\
Q3 & What of the following pieces of information are dangerous to reveal online?  \\
Q4  & Bad actors can use stolen personal information to do which of the following: \\
Q5 & What is a side channel attack?  \\
Q6  & Which of the following can give you a hint as to what a computer is doing?   \\
Q7 & What is an example of a common real-world side channel attack? \\
Q8  & How could you prevent an attacker from stealing a password by using a microphone to listen to keystrokes?\\
Q9 & Which of the following is a major use of hash functions? \\
Q10 & Which of the following is an important feature of a good hash function  \\
Q11 & When passing multiple items sequentially into a hash function, which item has the most influence on the output \\
Q12 & What is the term used when two or more inputs to a hash function generate the same output? \\
\hline
S1 & The lesson today increase my knowledge in the areas we covered. \\
S2  & The manner in which the material was presented helped me understand the material better\\
S3 & I enjoyed the manner in which the material was presented  \\
\end{tabular}
\caption{Questions used in our study instrument.  Questions Q1 through Q12
    appeared on both the pre-test and post-test.  Statements S1 through S3
    appeared only on the post-test.}
\label{fig:assessment}
\end{table}



\begin{table}
  \centering
  \scriptsize
  \begin{tabular}{c | l | l}
Question &  Correct on Pre-test   & Correct on Post-test \\
\hline
Q1   & 3 (60\%)  & 5 (100\%) \\
Q2   & 3 (60\%)  & 4 (80\%)  \\
Q3   & 5 (100\%) & 5 (100\%) \\
Q4   & 2 (40\%)  & 5 (100\%) \\
Q5   & 0 (0\%)   & 0 (0\%)   \\
Q6   & 5 (100\%) & 5 (100\%) \\
Q7   & 4 (80\%)  & 5 (100\%) \\
Q8   & 3 (60\%)  & 5 (100\%) \\
Q9   & 4 (80\%)  & 3 (60\%)  \\
Q10  & 4 (80\%)  & 4 (80\%)  \\
Q11  & 1 (20\%)  & 5 (100\%) \\
Q12  & 2 (40\%)  & 5 (100\%) \\
\end{tabular}

%\begin{tabular}{ c | l}
%Statement & Aggregate Responses \\
%\hline
%S1 &  Strongly Agree: 4 Agree: 1 \\
%S2 &  Strongly Agree: 4 Agree: 1 \\
%S3 &  Strongly Agree: 5 \\
%\end{tabular}}
\caption{ Aggregate results showing number of correct
  responses and percent of correct responses for
  each question on both our pre-test and post-test.}
%Rows S1, S2, and S3 report number of participants that selected the listed response.  Likert responses that were not chosen by any participant are omitted.
\label{fig:results}
\end{table}

\paragraph{Results}Overall, aggregate scores, shown in Table~\ref{fig:results},
increased across all categories on the post-test.
Scores for the social engineering and attacks on randomness sections
increased by 30\% while the side channel attacks section increased by
15\%.

The improvement in social engineering scores can be traced to
higher scores on
Q1, Q2, and Q4.
This indicates a better understanding of the technical details
around social engineering.
Encouragingly, the students were well aware of
what information should not be shared online suggesting they have been
warned about these dangers in other places.

The improvements in scores related to attacks on randomness
came largely
from more accurate responses to Q11 and Q12.
These questions are particularly important because
correct answers suggest the students have learned two
key properties of hash functions.
The fact Q9 scores improved a small amount and Q10 scores did not improve
means future iterations of the lesson should spend more time
explaining how these functions may be used in the real world.

The smallest improvement (15\%) came about in the side channel attacks
scores.  This is largely due to Q9 which proved difficult for students on
both the pre-test and post-test.  We believe this indicates a flaw in this
question and it will be reworked for future sessions.

In terms of student opinions about the lessons, Figure~\ref{fig:results}
shows the session scored very highly
on our Likert-scale statements.  A significant majority of students strongly
agreed that the lesson improved their skills in the covered topics.  A
similar majority felt the magic-based presentation was useful and
enjoyable.  A few students offered free response comments describing the
session as ``fun,'' ``entertaining,'' and ``interesting.''  One response
stated that the side-channel attack using upside-down cards
was particularly useful for helping them understand how such attacks are
structured.


%Comparing aggregate pre-test and post-test results show an improvement of
%AAA\%.  The majority of this improvement came from better scores on the
%hash function-related questions.  This improvement likely comes from our
%participant population's lack of experience with the area.  The details of
%hash functions are likely to be covered in undergraduate course work rather
%than high-school computer science classes.
%
%The next most-improved topic was side channel attacks.  Questions 6 and 7
%showed a marked improvement from AAA\% correct to BBB\% correct.  Answering
%these questions correctly requires students to understand the theory
%behind side
%channel attacks and apply it to several scenarios to identify the correct
%response.  As a result,  improved scores on these questions indicate deeper
%mastery of the material rather than simple memorization of a definition.
%
%The questions related to social engineering showed some improvement, though less
%than the other areas.  This is due to most participants correctly answering
%these questions on the pre-test.  It is likely students currently
%attending high school grew up in a time where education about scams and
%safety on the internet has been ongoing from an early age.


\subsection{Instructor Observations}

Based on instructor observation, student experience during
each session was overwhelmingly positive --
particularly in light of the teaching modality enforced by COVID-19
restrictions.
Of particular importance are two key indicators
of student engagement in a remote setting: camera usage and
responsiveness.  In both sessions a high majority of students used their
cameras when participating in a trick, asking, and answering questions
about the material.  Students remained attentive and focused throughout
 the lesson suggesting that our
style of presentation can help maintain attention during a moderately long,
technical lecture.

Student responsiveness during both sessions was very high.
They were eager to analyze the
tricks and put forward hypotheses about how they worked.
When prompted to
expand on certain aspects of the material, students quickly volunteered and
offered responses indicating a growing understanding of the topic.
Further, students readily volunteered to participate in the tricks in spite
of the unusual teaching modality created by a remote presentation,
along with the general discomfort of being in a
prominent position ``at the front of class.''

After the session,
unprompted verbal comments attest to the lesson's effectiveness.
Students volunteered that they enjoyed the
lesson and found the magic tricks aided their understanding of the
material.  These comments support the appeal of our lesson's approach and
encourage its use in this and other subject areas.
