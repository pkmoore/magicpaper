\section{Study Instrument and Evaluation}
\label{SEC:evaluation}

\subsection{Method}

The goal of our study was to judge how effective a non-traditional approach
could be in teaching novices about our selected attacks.  To do so,
we put together a 90 minute Zoom session to be presented at a summer
computer science workshop for high school students.  The session was
offered twice during the three week workshop, with the lead investigator
serving as the presenter and magician.  It consisted of three phases, one
for each topic, and each consisting of a performance of the topic's magic
trick, followed by a detailed discussion of the topic using the magic trick
as a scaffold.  A total of AAAAAA students attended the two sessions.

To measure improvements in participants' mastery of the subject, we
designed an assessment instrument (shown in figure~\ref{fig:assessment})
consisting of
12 multiple choice questions (4 for each topic),
3 Likert-scale survey statements,
and a free response comments section.
All of these materials were reviewed and approved by our university's
Institutional Review Board and, though attendance was open to all, we only
used data from students who had given us their consent to be part of the
study.

Participants completed the 12 multiple choice questions twice,
once before our lecture to gather a baseline of
participants' existing knowledge of the material,
and again at the conclusion of our
lecture as a summative assessment to determine if student mastery improved.
Both assessments used the same questions.
After taking part in the workshop, participants also evaluated the session
itself by offering opinions on whether or not they found our lecture
enjoyable and engaging.
The summative assessment was given at the conclusion of the lesson
with the option to complete it after the lesson in their own time.
BBB students completed our formative and summative assessments and CCC
completed both of these components in addition to our survey statements.

\begin{figure*}[htb!]
  \scriptsize
  \begin{tabular}{c | p{8cm} | c | p{8cm}}
   & Question Text  &     & Question Text\\
\hline
Q1 & Which of the following is the best definition of social engineering  & Q2  & The act of creating a scenario in order to extract information is called:   \\
Q3 & What of the following pieces of information are dangerous to reveal online?  & Q4  & Bad actors can use stolen personal information to do which of the following: \\
Q5 & What is a side channel attack?  & Q6  & Which of the following can give you a hint as to what a computer is doing?   \\
Q7 & What is an example of a common real-world side channel attack?  & Q8  & How could you prevent an attacker from stealing a password by using a microphone to listen to keystrokes?\\
Q9 & Which of the following is a major use of hash functions? & Q10 & Which of the following is an important feature of a good hash function  \\
Q11 & When passing multiple items sequentially into a hash function, which item has the most influence on the output & Q12 & What is the term used when two or more inputs to a hash function generate the same output? \\
\hline
S1 & The lesson today increase my knowledge in the areas we covered. & S2  & The manner in which the material was presented helped me understand the material better\\
S3 & I enjoyed the manner in which the material was presented &     &              \\
\end{tabular}
\caption{Questions used in our study instrument.  Questions Q1 through Q12
    appeared on both the pre-test and post-test.  Statements S1 through S3
    appeared only on the post-test.}
\label{fig:assessment}
\end{figure*}


\subsection{Results}

Comparing aggregate pre-test and post-test results show an improvement of
AAA\%.  The majority of this improvement came from better scores on the
hash function-related questions.  This improvement likely comes from our
participant population's lack of experience with the area.  The details of
hash functions are likely to be covered in undergraduate course work rather
than high-school computer science classes.

The next most-improved topic was side channel attacks.  Questions 6 and 7
showed a marked improvement from AAA\% correct to BBB\% correct.  Answering
these questions correctly require students to understand theory behind side
channel attacks and it to several scenarios to identify the correct
response.  As a result,  improved scores on these questions indicate deeper
mastery of the material rather than simple memorization of a definition.

The questions related to social engineering showed improvement, though less
than the other areas.  This is due to most participants correctly answering
these questions on the pre-test.  It is likely because students currently
attending high school grew up in a time where education about scams and
safety on the internet has been ongoing from an early age.


\subsection{Instructor Observations}

Student experience during each session was overwhelmingly positive --
particularly in light of the unusual teaching modality.
Of particular importance are two key indicators
of student engagement on a remote setting: camera usage and
responsiveness.  In both sessions a high majority of students used their
cameras when participating in a trick, asking, and answering questions
about the material.  Students appeared attentive and focused on the lesson.
Their focus did not wane as the lesson progressed suggesting that our
style of presentation can help maintain attention during a moderately long,
technical lecture.

Student responsiveness during both sessions was very high.
They were eager to analyze the
tricks and put forward hypotheses about how they worked.
When prompted to
expand on certain aspects of the material students quickly volunteered and
offered responses indicating a growing understanding of the topic.
Further, students readily volunteered to participate in the tricks in spite
of the unusual teaching modality and general discomfort of being in a
prominent position ``at the front of class.''

After the session,
unprompted verbal comments support support the lesson's effectiveness.
Students volunteered that they enjoyed the
lesson and found the magic tricks aided their understanding of the
material.  These comments support the appeal of our lesson's format and
encourage its use in this and other subject areas.
