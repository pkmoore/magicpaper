\section{Study Instrument and Evaluation}
\label{SEC:evaluation}

\paragraph{Method}The goal of our study was to judge how effective a non-traditional approach
could be in teaching novices about our selected attacks. To do so, we prepared a 90-minute Zoom session and offered it as a optional class during
a remote-learning computer science summer camp for high school
students. Using this particular format was not a choice, but rather a workaround made necessary when COVID-19 restrictions prevented the summer camp from being held live. We discuss the impact of this on our study later in this section.


We measured improvement in mastery
of the material
through
an  assessment (a portion of which is shown in Table~\ref{fig:assessment})
consisting of
12 multiple choice questions (4 for each topic),
3 Likert-scale survey statements,
and a free response section.
The assessment was conducted
online, before the lesson to
generate a baseline and then again after the lesson to measure improvement
and gather student opinions.  We purposely avoided collecting demographic information
due to the heightened privacy concerns inherent in working with high school students.


%All of these materials were reviewed and approved by our university's
%Institutional Review Board and, though attendance was open to all, we only
%used data from students who had given us their consent to be part of the
%study.


%They were asked to complete it once

%completed the 12 multiple choice questions twice,
%once before our lecture to gather a baseline of
%participants' existing knowledge of the material,
%and again at the conclusion of our
%lecture as a post-test to determine if student mastery improved.
%Both assessments used the same questions.
%The post-test was handed out at the conclusion of the lesson
%and students were asked
%to complete it in their own time.

%to be presented at a summer
%computer science workshop for high school students.  The session was
%offered twice during the three week workshop, with the lead investigator
%serving as the presenter and magician.  It was divided into three phases,
%one
%for each topic,
%and each phase consisted of a performance of the topic's magic
%trick, followed by a detailed discussion that used the magic trick
%as a scaffold.  A total of 15 students attended the two sessions.


%Given the overwhelmingly
%positive feedback we received from our session attendees, we believe the
%decreased number of participants completing the post test can likely
%be attributed
%to the awkwardness of an interactive session online
%rather than deficiencies in
%the session itself.


\begin{table}[h]
  \scriptsize
  \begin{tabular}{c | p{6cm} | l | l}
  & Question Text & correct on pre-test & correct on post-test    \\
\hline
Q1 & Which of the following is the best definition of social engineering & 3 (60\%) & 5 (100\%)\\
Q2  & The act of creating a scenario in order to extract information is called:  & 3 ( 60\%) & 4 (80\%) \\
Q3 & What of the following pieces of information are dangerous to reveal online?  & 5 (100\%) & 5 (100\%)\\
Q4  & Bad actors can use stolen personal information to do which of the following: & 2 (40\%) & 5 (100\%)\\
Q5 & What is a side channel attack?  & 0 (0\%) & 0 (0\%) \\
Q6  & Which of the following can give you a hint as to what a computer is doing?   & 5 (100\%) & 5 (100\%) \\
Q7 & What is an example of a common real-world side channel attack? & 4 (80\%) & 5 (100\%) \\
Q8  & How could you prevent an attacker from stealing a password by using a microphone to listen to keystrokes? & 3 (60\%) & 5 (100\%)\\
Q9 & Which of the following is a major use of hash functions? & 4 (80\%) & 3 (60\%) \\
Q10 & Which of the following is an important feature of a good hash function  & 4 (80\%) & 4 (80\%) \\
Q11 & When passing multiple items sequentially into a hash function, which item has the most influence on the output & 1 (20\%) & 5 (100\%) \\
Q12 & What is the term used when two or more inputs to a hash function generate the same output? & 2 (40\%) & 5 (100\%) \\
\end{tabular}
% \begin{tabular}{c | l | l }
%Question &  Correct on Pre-test   & Correct on Post-test \\
%\hline
%Q1   & 3 (60\%)  & 5 (100\%) \\
%Q2   & 3 (60\%)  & 4 (80\%)  \\
%Q3   & 5 (100\%) & 5 (100\%) \\
%Q4   & 2 (40\%)  & 5 (100\%) \\
%\hline
%Q5   & 0 (0\%)   & 0 (0\%)   \\
%Q6   & 5 (100\%) & 5 (100\%) \\
%Q7   & 4 (80\%)  & 5 (100\%) \\
%Q8   & 3 (60\%)  & 5 (100\%) \\
%\hline
%Q9   & 4 (80\%)  & 3 (60\%)  \\
%Q10  & 4 (80\%)  & 4 (80\%)  \\
%Q11  & 1 (20\%)  & 5 (100\%) \\
%Q12  & 2 (40\%)  & 5 (100\%) \\
%\end{tabular}
\caption{Question text and aggregate scores for each assessment question. Q1-Q4 covered
  social engineering, Q5-Q8 covered side channel attacks, and Q9-Q12
  covered attacks on randomness.}
\label{fig:results}
\end{table}

\paragraph{Results}
Aggregate scores increased across all categories on the post-test.
The results in Table~\ref{fig:results} show scores for the social engineering and attacks on randomness sections
increased by 30\% while the side channel attacks section increased by
15\%.
The improvement in social engineering scores can be traced to
higher scores on
Q1, Q2, and Q4,
indicating a better understanding of the topic.
Smaller improvements on Q5 and Q9 suggest a need to improve the lesson in
these specific areas, particularly furthering our broad overview of side channel attacks
and real-world use cases for hash functions.
Accurate responses to Q11 and Q12
suggest the lesson was an effective scaffold for teaching two
key properties of hash functions.


Student offered very positive opinions about the lessons either agreeing or strong agreeing
that the lesson improved their skills in the covered topics
and was enjoyable.
More than half of students offered free response comments describing the
session as ``fun,'' ``entertaining,'' and ``interesting.''  These opinions reinforce instructor observations made during the sessions.  A significant majority of students kept their cameras on and participated by asking or answering questions about the material -- two key indicators of engagement during remote instruction.  

%\paragraph{Instructor Observations}
%The session's instructor observed that student experience during
%the session was overwhelmingly positive considering
%the difficulties imposed by a remote teaching modality required by COVID-19
%restrictions.
%In a remote setting, camera usage
%and responsiveness are two key indicators of student engagement.
%In both sessions a significant majority of students kept their cameras on
%when participating in a trick, and asking or answering questions
%about the material.
%%Students remained attentive and focused throughout
%% the lesson suggesting that our
%%style of presentation can help maintain attention during a moderately long,
%%technical lecture.
%Students were also
%eager to analyze the
%tricks, put forward hypotheses about how they worked,
%and expand on important aspects of the material.
%Further, students readily volunteered to participate in spite
%of the unusual teaching modality created by a remote presentation,
%along with the general discomfort of being in a
%prominent position ``at the front of class.''
%After the session,
%several
%made
%unprompted verbal statements about how much they enjoyed the
%lesson and how the magic tricks aided their understanding of the
%material.


\paragraph{Limitations}

Conducting this work was not without its share of challenges.
COVID-19, a remote modality, and difficulties handling consent forms
reduced participation in our sessions.
Of the 15 students that attended our session, 10 agreed to participate in
the study and 5 completed it.
The fall off in study completion can likely be attributed to
an inability to
do the assessment in person and to
follow up about the post-test.
It was simply too easy for students to sign off and forget
to respond to the post test.
This limited completion rate prevents us from making strong
statistical claims about the effectiveness of our lessons.
Instead, we must rely on the individual score improvements and opinions
of a handful of students as well as their positive remarks about
our sessions.



\paragraph{Future Work}
Encouraged by the positive results from this evaluation,
we plan to continue refining and presenting our lesson.
In the coming months we intend to carry out larger, in-person sessions
with groups of undergraduate computer science students.
Our hope is to demonstrate that our method remains effective
when teaching older students with more computer science experience.
