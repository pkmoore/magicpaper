\section{Trick Discussion}
\label{SEC:background}

In this section we discuss the details of each of the tricks we used in our
lesson by first describing the intended pedagogical goal of the trick and then
how the trick works from both the participants' and the magician's
perspectives.  We also include variations for each trick that can improve the
trick's impact or make it easier to perform under certain circumstances.

\section{Social Engineering}

The purpose of this trick is to increase students' awareness of social
engineering by exposing them to a simulated attack in the form of a magic trick.
This is done by engineering the card trick to give the magician multiple
opportunities to ask a student to provide personal information, such as their
full name or place of birth, that may be used for identity theft.
At the end of the trick, the magician reveals that this information collection
was the real purpose of the trick.

\subsection{From the Audience's Perspective}

The magician begins this trick by requesting a student volunteer and stating
that there is a way to form a psychic bond between the student and a deck of
cards.  This is initially achieved by having the student shuffle approximately
half of the cards a few times while the magician shuffles the other half.  Once
this shuffling is complete, the two halves of the deck are combined by placing
one on top of the other.
The magician then asks the student
a few more questions to further ``attune'' them to the deck.
First they are asked for their birth year and their response
is used to select one red card and one black card from the deck
with each with a numeric value equal to one of the last two digits of their
birth year.
Next, the magician asks for the student's birth month and similarly selects
a red card with a numeric value equal
to their birth month (using the jack and queen for November and December
respectively).
Finally, the magician asks the student
for their birth day and select a black card with a numeric value
equal to the second digit in this day.
The magician lays these cards out on the table and asks the student to select
one red card and one black card with which they feel most ``attuned''.
The unchosen cards are returned to the middle
of the deck face up.

At this point, it is time to test the student's psychic powers by asking them to
guess the color of each card in the deck.  Starting with the top of the deck,
the student provides a guess of either ``red'' or ``black'' and the magician
records their guess by placing the card face down on top of the card of the same
color already on the table.  This forms two piles of cards.  When the half way
point of the deck is reached, the two face up cards are placed on top of the
pile of the opposite color.  This switches which pile each guess
will be placed in going forward.  The student continues providing guesses until
there are no cards remaining.  At this point the magician turns over the face
down cards to reveal that the student has correctly guessed {\textbf every}
card's color.

The magician now has an opportunity to further illustrate the student's psychic
link to the deck of cards by using said link to read the student's mind.
First, the cards are collected and the student is asked to
re-shuffle them a few time and cut the deck into two piles.  The magician
selects one of the piles and reveals the bottom card to only the student and the
audience.  This pile is placed on top of the other pile perpendicularly so that
the students card remains accessible.  The magician then asks the student to
make a series of statement pairs with one being true and one being false.  For
example,  the magician may ask the student to state  ``I was born in
\textit{blank},'' with blank being replaced by their birth city, and ``I was
born in Moscow, Russia,'' (assuming the student was \textit{not} born in Moscow,
Russia).  A similar process is used to have the student make true/false
statement pairs about where they attended school earlier in their life, the name
of their childhood best friend, and other similar details about their life.

After the magician as elicited a number of statements sufficient to hone in on
the student's psychic link, a second set of true/false statement pairs are used
to figure out the student's card.  This is done by asking the student to make
statements pairs like ``My card is red'' and ``My card is black.''  Similar
pairs regarding the card's suit and value allow the magician to eventually zero
in on the student's card.

At this point, the trick takes a turn when the magician presents a spreadsheet
to the student and the audience containing their responses to the questions
asked in the beginning of the trick revealing that the trick was a deception to
collect personal information.  This reveal is followed by discussion about what
types of information should can be used in identity theft attempts should they
be made public.


\subsection{Behind the Scenes}

Begin with a deck where the cards have been separated into red and black halves.

\subsection{Variations}

...

\section{Side Channel Attacks}

\subsection{From the Audience's Perspective}

\subsection{Behind the Scenes}

\subsection{Variations}


\section{Cryptographic Hash Functions}

\subsection{From the Audience's Perspective}

\subsection{Behind the Scenes}

\subsection{Variations}
