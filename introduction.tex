\section{Introduction}
\label{SEC:introduction}

When teaching technical topics
in introductory courses,
it can be challenging
to present information
in a way that makes sense 
for students of varying
experience levels
or educational backgrounds.
This is particularly true
for information security classes
where an adversarial mindset
is required
to fully appreciate the impact of the attacks and the effort required to
defend against them.
Thinking in this way
may not come naturally
to many students,
as evidenced
by the continuing success of phishing attacks
and scams.
What is needed
is a way
to relate information security concepts
to novice students
in a manner
that is engaging enough
to build an appreciation for the material,
and relatable enough so the students do not feel lost.
To accomplish this, we look towards a pedagogical technique known as scaffolding, which teachers for employed for many years to help students build
new knowledge on existing knowledge.
As explained in one source,
``students are escorted and monitored through learning
activities that function as interactive conduits to get
them to the next stage~\cite{raymond2000}.'''

In this paper, 
we employ card magic as a scaffolding device in a series of lessons that teach how three types
of attacks -- social engineering, side channel attacks, and attacks on
randomness -- work in the real world. In doing so, we build on the
success other computer science researchers have had 
in using card magic
to explain computer science concepts by allowing the instructor to simulate “attacks” through a non-technical, commonplace activity. In doing so, these lessons can help students get into the security mindset by offering them opportunities
to safely interact within these attack scenarios.
Better yet, the tricks are simple enough for any instructor to quickly learn and use.
Video tutorials for each of our tricks are available at: \textit{URL removed
for blinding purposes}.

To test the effectiveness of our lessons, we presented them to
a group of high school students
attending a computer science summer program.
Based on pre- and post- tests covering the
subject matter, our lessons were effective 
in improving the participants' mastery of the material,
with scores
increasing by between 15\% and 30\% for each category. In addition, based on both an opinion survey and the observations of the instructor, 
participants found the lessons engaging, age appropriate, and helpful in understanding the concepts.

The main contributions of this work are as follows:

\begin{itemize}

\item{We create a lesson plan that includes three easy-to-perform magic
  tricks in which the instructor can simulate three specific types of attacks}

\item{We test the effectiveness of these lessons by presenting them to
  a group of 15 high school students
  participating in a summer workshop and assessing their comprehension.}

\item{We note an improved ability to answer questions related to the
  attacks following our lesson,
    as judged by pre- and post-test evaluations.}
\end{itemize}

