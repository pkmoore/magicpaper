\section{Introduction}
\label{SEC:introduction}

When teaching technical topics
in introductory courses,
it can be challenging
to present information
in a way that is approachable
for students of varying
experience levels
or educational backgrounds.
This is particularly true
for information security
where an adversarial mindset
is required
to fully appreciate the impact of attacks and the effort required to
defend against them.
Thinking in this way
may not come naturally
for many students
as evidenced
by the success of phishing attacks
and scams in the real world.
What is needed
is a way
to relate information security concepts
to novice students
in a manner
that is engaging enough
to build an appreciation for the material,
and relatable enough that the students do not feel lost.

In this paper
we develop three card-magic-based lessons
for introductory information security courses
that take inspiration
from the success others have had
in using magic
to explain computer science concepts.
The specific trick and presentation technique
for each lesson was designed to emphasize the primary
features of the concept being taught,
which for our study were three types of attacks:
social engineering,
side channel attacks,
and attacks on randomness.
The lessons offered natural opportunities
for direct student interaction direct student interaction.
None of the tricks require advanced sleight of hand
or card manipulation,
making it easy any instructor to quickly learn and use them.


We tested our premise that concepts taught using the familiar approach
of card tricks would improve attentiveness and comprehension
by conducting them for high school students
attending a computer science summer program.
To gauge the lessons'
effectiveness we had students complete pre- and post- tests covering the
material being taught.  These tests showed that our lessons were highly
effective in increasing the participants mastery of the material with scores
increasing by AAA\% on average.
We also had our participants complete an opinion survey to get an idea of
whether or not the lessons were engaging and age appropriate.  Once again, our
lessons were rated very highly,
with the majority responding that this style of lesson made material
more easily understandable.
We further augment these responses
with a report of observations
from the instructor,
who found students were engaged by the lessons and enjoyed their interactive
aspects.


The main contributions in this work are as follows:

\begin{itemize}

\item{We create a lesson plan that includes three easy-to-learn magic
  tricks based on three specific types of attacks,
    presented in order of increasing complexity.}

\item{We test these lessons with a group of AAA high school students
  participating in a summer workshop.}

\item{We note an improved ability to answer questions related to the
  attacks following our lesson as well as high student opinion of the
    lesson as determined by a survey and instructor observations.}

\end{itemize}
