\section{Introduction}
\label{SEC:introduction}

When teaching technical topics
in introductory courses
in can be challenging
to present information
in a way that is approachable
for students of varying
experience levels
or educational backgrounds.
This is particularly true
for information security
where an adversarial mindset
is required
to fully appreciate the impact of attacks and the effort required to
defend against them.
Thinking in this way
may not come naturally
for many students
as evidenced
by the success of fishing attacks
and scams in the real world.
What is needed
is a way
to relate information security concepts
to novice students
in a manner
that is engaging enough
to build an appreciation for the material
and relatable enough that they remain approachable.

In this paper
we develop three card-magic-based lessons
for introductory information security courses
taking inspiration
from the success others have had
in using magic
to explain computer science concepts.
Our lessons cover social engineering,
side channel attacks,
and {\textbf ...}.
The specific trick and presentation technique
for each lesson was designed to emphasize the primary
features of the concept being taught
while fostering direct student interaction.
None of the tricks require advanced slight of hand
or card manipulation making
it easy for instructors to quickly learn and apply them.


We evaluated our lessons by conducting them for high school students
who were attending a computer science summer program.  To guage the lessons'
effectiveness we had students complete pre- and post- tests covering the
material being taught.  These tests showed that our lessons were highly
effective in increasing the participants mastery of the material with scores
increasing by AAA\% on average.
We also had our participants complete an opinion survey to get an idea of
whether or not the lessons were engaging and age appropriate.  Once again, our
lessons rated very highly amongst students with the large majority responding
that the style of lesson made material more easily understandable.  Finally, we
further augment these responses with a report of instructor observations that
found students were engaged by the lessons and enjoyed their interactive
aspects.


The main contributions in this work are as follows:

\begin{itemize}

\item{We describe three magic tricks geared toward teaching
information security concepts like social engineering, side channel attacks, and
\textbf{.....}}

\item{We conduct study showing that these magic tricks improved students'
comprehension of the subject matter.}

\item{We relate the positive experiences of participants through instructor
observations and an opinion survey}

\end{itemize}
