\section{Introduction}
\label{SEC:introduction}

When teaching technical topics
in introductory courses,
it can be challenging
to present information
in a way that is approachable
for students of varying
experience levels
or educational backgrounds.
This is particularly true
for information security classes
where an adversarial mindset
is required
to fully appreciate the impact of attacks and the effort required to
defend against them.
Thinking in this way
may not come naturally
to many students,
as evidenced
by the continuing success of phishing attacks
and scams in the real world.
What is needed
is a way
to relate information security concepts
to novice students
in a manner
that is engaging enough
to build an appreciation for the material,
and relatable enough that the students do not feel lost.

In this paper
we develop three card-magic-based lessons
for introductory information security courses
that take inspiration
from the success others have had
in using magic
to explain computer science concepts.
The key to teaching these unknown concepts
lies in a common pedagogical technique
known as scaffolding.
Scaffolding is a process that lets students build
new knowledge on existing knowledge.
As explained in one source,
``students are escorted and monitored through learning
activities that function as interactive conduits to get
them to the next stage~\cite{raymond2000}.''
In the field of computer science,
several researchers have reported success in using card
tricks—something most people
have seen either at a birthday party
or on television— to teach complex concepts.
We decided to employ this technique to teach how three types
of attacks -- social engineering, side channel attacks, and attacks on
randomness -- work in the real world.
The specific trick and presentation technique
for each lesson emphasized the primary
features of the concept being taught.
The lessons also offered natural opportunities
for direct student interaction.
Better yet, none of the tricks require advanced sleight of hand
or card manipulation,
making it easy for any instructor to quickly learn and use them.
Video tutorials for each of our tricks are available at: \textit{URL removed
for blinding purposes}.


We tested the effectiveness of our lessons on
a group of high school students
attending a computer science summer program.
We had students complete pre- and post- tests covering the
material being taught.  These tests showed that our lessons were highly
effective in improving the participants mastery of the material,
with scores
increasing by between 15\% and 30\% for each category.
We also had our participants complete an opinion survey to get an idea of
whether or not the lessons were engaging and age appropriate.  Once again, our
lessons were rated very highly,
with the majority responding that this style of lesson made material
more easily understandable.
We further augment these responses
with observations
from the instructor,
who found students were engaged by the lessons and enjoyed their interactive
aspects.


The main contributions in this work are as follows:

\begin{itemize}

\item{We create a lesson plan that includes three easy-to-perform magic
  tricks based on three specific types of attacks,
    and presented in order of increasing complexity.}

\item{We test the effectiveness of these lessons with
  a group of AAA high school students
  participating in a summer workshop.}

\item{We note an improved ability to answer questions related to the
  attacks following our lesson,
    as judged by pre- and post-test evaluations.  Based on post-session
    surveys and instructor observations, the non-traditional approach was
    also engaging and effective.}

\item{We make available tutorial videos of our tricks so that other
  instructors can incorporate them into their own lessons}

\end{itemize}
