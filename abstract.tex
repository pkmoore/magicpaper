\begin{abstract}

One of the main challenges
in designing lessons
for an introductory
information security class
is how to present new technical concepts
in a manner comprehensible to students
with widely different backgrounds. A non-traditional approach
can help students engage with the material
and master these unfamiliar ideas.
We have devised a series of lessons that teach important information security
topics, such as social engineering,
side-channel attacks,
and
attacks on randomness
using card magic.
Each lesson
centers around a card trick
that allows the instructor
to simulate the described attack
in a way that makes sense even for those who have no prior technical
background.
In this paper,
we describe our experience using these lessons in teaching
cybersecurity topics
to high school students with limited computer science education.
Students were assessed
before and after the demonstration
to gauge their mastery of the material,
and their
opinions on each lesson.
Our results indicate that students enjoyed the lesson
and their pre- and post-test scores improved by between 15\% and 30\%.


\end{abstract}

%Hand-waving away the technical background behind these concepts is bad...
%
%...and undermines student's motivation to learn and understand the material.
%
%Bypassing the need for technical explanation while preserving the main
%  focus of the lesson.
