\begin{abstract}

One of the main challenges
in designing lessons
for an introductory
information security class
is how to present new technical concepts
in a manner comprehensible to students
with widely different backgrounds.
In situations like these, a non-traditional approach
can help students
engage with the material
and, in doing so,
master
these unfamiliar ideas.
We have devised a series of lessons to teach important information security
topics such as social engineering,
side-channel attacks,
and
attacks on randomness
by using card magic
to illustrate the central idea.
Each lesson
centers around a card trick
that emphasizes the lesson's objectives
and offers the opportunity
for student participation.
In this paper,
we describe our experience using these lessons to teach
high school students with limited computer science background
the information security topics mentioned above.
Students were assessed
before and after the demonstration
to gauge their mastery of the material.
Students also
completed a short survey designed to gather their
opinions of each lesson.
In addition to very positive student interactions
during the lesson,
our formative and summative assessments
showed that student scores improved by XX\%
after attending our demonstration.
Students reported that they enjoyed the lesson,
found the card magic a useful tool,
and reported that the demonstration
improved their mastery of the material.

\end{abstract}

%Hand-waving away the technical background behind these concepts is bad...
%
%...and undermines student's motivation to learn and understand the material.
%
%Bypassing the need for technical explanation while preserving the main
%  focus of the lesson.
