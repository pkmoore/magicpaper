\begin{abstract}

One of the main challenges faced when designing lessons for an introductory
computer science class
is presenting technical topics while overcoming students' varying
backgrounds.
In situations like these, a non-traditional approach
can help students master important concepts
and foster engagement with the material.
We have devised a series of lessons on important information security
topics such as social engineering, side-channel attacks, and secure use
of hashing algorithms that use card magic to aid to help explain the details.
Each lesson centers around a card trick that emphasizes the lesson's
objectives and offers the opportunity for active learning through
participation.
In this paper,
we describe our experience using these lessons to teach
high school students with limited computer science background about the
above information security topics.
After each lesson, students were given a quiz
to assess their mastery of the material.
We further had students complete a short survey designed to gather their
opinions of each lesson.

\end{abstract}

%Hand-waving away the technical background behind these concepts is bad...
%
%...and undermines student's motivation to learn and understand the material.
%
%Bypassing the need for technical explanation while preserving the main
%  focus of the lesson.
