\section{Related Work}
\label{SEC:related-work}

\subsection{Card Magic in Computing Education}
There is some history of using card magic to convey or explain computer science
concepts.

\paragraph{Error Correction and Parity}
Early work under the ``Computer Science Unplugged'' umbrella by Bell et al. uses
a 5 by 5 grid of cards in an exercise designed to teach about parity and
error correction~\cite{bell2009computer}~\cite{csunplugged}.
Greenberg et al. expand upon this style of trick by creating more advanced
versions with larger grids.  Some new versions use additional computer science
techniques to increase the size of the grid that can be handled by a human
magician.  Others rely on software assistance to handle more complex
computation~\cite{Greenberg2017}~\cite{Greenberg2018}.

\paragraph{Algorithm Analysis and Design}
Ferreira et al. use ``self-working'' card trick called ``Are You Psychic''
as a medium to explain topics
like problem decomposition, pre- and post- conditions, and
invariants~\cite{ferreira2014magic}.
The trick's relevance to each lesson is illustrated by
mapping each of its steps.
onto a formal description of the trick's algorithm.

\paragraph{Multi-trick Magic Shows}
Garcia et al. have produced three papers describing a wide variety of magic
tricks along the computer sciences concepts they can help
teach~\cite{garcia2012demystifying}
~\cite{garcia2013demystifying}
~\cite{garcia2016demystifying}.
Similarly, Curzon et al. found success explaining computer
science concepts to younger students using magic shows~\cite{Curzon2008}.

\subsection{Demonstrations as Scaffolding}

Our use of card tricks as scaffolding when introducing new concepts well
supported by education literature.

\paragraph{Magic Tricks as Scaffolding}
The purpose of our magic tricks and demonstrations is to introduce
information security concepts in such a way that students from varying
backgrounds are able to understand them.  This methodology has a long history of
support starting with Vygotsky who described the need to bring topics into a
student's ``Zone of Proximal Development.~\cite{vygotsky1978}''
A common approach to performing this migration is to provide assistance in the
form of ``scaffolding'' that can assist students in mastering material that may
be beyond their reach~\cite{wood1976role}.  We believe our results show that our
demonstrations are effective scaffolds for introducing novel security concepts
to our participants.

\paragraph{Use of Scaffolds in Computer Science education}
The use of scaffolds in Computer Science education is well supported. In a
meta-analysis from 2019 Szabo et
al. identify 1283 papers containing scaffolding related
content~\cite{szabometa}.  Work by Vanderyde et al. argues for greater use of scaffolding and
engaging teaching practices in order to improve learning in light of increased
Computer Science enrollment and diversity in Computer Science
programs~\cite{vanderhydescaffolding}.  Stanier discusses further success in
using scaffolds to introduce new topics in a higher education
environment~\cite{stanierhighered}.
