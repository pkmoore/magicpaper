\section{Related Work}
\label{SEC:related-work}

The idea of scaffolding is to provide a bridge to assist students
in mastering material that may be beyond their reach~\cite{wood1976role}
by bringing it into what psychologist Lev Vygotsky described as a student's
``Zone of Proximal Development~\cite{vygotsky1978}.''  Given the complexity of computer science
topics, it is not surprising that researchers have attempted to ``scaffold''
these concepts from a familiar base.  In a meta-analysis from 2019, Szabo et al.
identified 1283 papers in the field that contain scaffolding related
content~\cite{szabometa}.
Vanderyde et al. argues for greater use of scaffolding and engaging teaching
practices in order to improve learning in light of increased computer science
enrollment and diversity in computer science
programs~\cite{vanderhydescaffolding}.  Stanier discusses
further success in using scaffolds to introduce new topics in a higher education
environment~\cite{stanierhighered}.
We believe the results show that our demonstrations work as
effective scaffolds for introducing novel security concepts to novices.

Other researchers have already integrated card tricks into lesson plans about
computer science concepts, such as using parity bits to detect unintended bit
flips, a central technique in error detection and correction.  Bell et al. uses a
5 by 5 grid of cards in an exercise that transforms the concepts into a hands-on
exercise~\cite{bell2009computer, csunplugged}.
The exercise improves students' understanding by allowing them to
both generate parity errors and participate in their detection.  Greenberg et
al. were able to create more advanced versions using larger grids.  Other versions
of this activity rely on software assistance to handle more complex
computations~\cite{Greenberg2017, Greenberg2018}.

In Ferreria et al. a ``self-working'' card trick called ``Are You Psychic?''
is used
as a medium to explain topics in algorithm analysis and design, such as problem
decomposition, pre- and post- conditions, and
invariants~\cite{ferreira2014magic}.
The trick's relevance
to each lesson is illustrated by mapping each of its steps onto a formal
description of the algorithm.  Garcia et al. have produced three papers describing
a wide variety of magic tricks, along with the computer science concepts they
can help teach~\cite{garcia2012demystifying,
garcia2013demystifying,
garcia2016demystifying}.
Their goal was to help students construct a mental model of how
a computer actually works, which they argue is an essential component of any
introductory course.  Similarly, Curzon et al. found success explaining computer
science concepts to younger students using magic shows~\cite{curzon2008engaging}.


%The card tricks and lessons we developed were inspired by previous research
%showing the effectiveness of magic to convey or explain computer science
%concepts.
%%We summarize these earlier initiatives in
%%Section~\ref{sec:magicineducation}, followed by a look at how card
%%demonstrations work as scaffolding devices in
%%Section~\ref{sec:demosasscaffolding}.
%
%\paragraph{Card Magic in Computing Education:}
%Given the complexity of many computer science topics, it's perhaps not
%surprising that the use of card magic to explain key concepts is already an
%established practice.  Below are a few examples from which we took inspiration
%when creating our information security-focused lessons.
%
%\paragraph{Error Correction and Parity}
%While using parity bits to detect unintended bit flips is a central technique
%in error detection and correction, it can sometimes seem unusual or unintuitive
%to novice computer scientists.
%The ``Computer Science Unplugged'' umbrella of teaching resources offers
%several examples.  Bell et al. uses
%a 5 by 5 grid of cards in an exercise designed to teach about parity and
%error correction by transforming the concepts into a hands on
%exercise~\cite{bell2009computer}~\cite{csunplugged}.  The exercise improves
%student's understanding by allowing them to both generate parity errors and
%participate in their detection.
%Greenberg et al. expand upon this style of trick by creating more advanced
%versions using larger grids.  Some new versions also use
%more complex error detection
%strategies to increase the size of the grid that can be handled by a human
%magician.  Others rely on software assistance to handle more complex
%computation~\cite{Greenberg2017}~\cite{Greenberg2018}.
%
%\paragraph{Algorithm Analysis and Design}
%Ferreira et al. use a ``self-working'' card trick called ``Are You Psychic''
%as a medium to explain topics
%like problem decomposition, pre- and post- conditions, and
%invariants~\cite{ferreira2014magic}.
%The trick's relevance to each lesson is illustrated by
%mapping each of its steps
%onto a formal description of the trick's algorithm.
%
%\paragraph{Multi-trick Magic Shows}
%Garcia et al. have produced three papers describing a wide variety of magic
%tricks, along with the computer sciences concepts they can help
%teach~\cite{garcia2012demystifying}
%~\cite{garcia2013demystifying}
%~\cite{garcia2016demystifying}.
%Their goal was to help students construct a mental model of how a computer
%actually works, which they argue is an essential component of any introductory
%course.
%Similarly, Curzon et al. found success explaining computer
%science concepts to younger students using magic shows~\cite{Curzon2008}.
%
%\paragraph{Demonstrations as Scaffolding}
%Our use of card tricks as scaffolding when introducing new concepts is also well
%supported by education literature.
%
%\paragraph{Why Magic Tricks work as Scaffolding}
%The purpose of our magic tricks and demonstrations is to introduce
%information security concepts in such a way that students from varying
%backgrounds are able to understand them.  This methodology has a long history of
%support, starting with psychologist Lev Vygotsky
%who described the need to bring topics into a
%student's ``Zone of Proximal Development.~\cite{vygotsky1978}''
%A common approach to performing this migration is to provide assistance in the
%form of ``scaffolding'' that can assist students in mastering material that may
%be beyond their reach~\cite{wood1976role}.  We believe our results show that our
%demonstrations are effective scaffolds for introducing novel security concepts
%to our participants.  Scaffolding is widely used in other areas and is something
%the information security community should consider employing to a larger degree.
%
%\paragraph{Use of Scaffolds in Computer Science Education}
%Perhaps because its core concepts can appear daunting to new students,
%the use of scaffolds in Computer Science education is well supported. In a
%meta-analysis from 2019 Szabo et
%al. identify 1283 papers in the field that contain scaffolding related
%content~\cite{szabometa}.
%Vanderyde et al. argues for greater use of scaffolding and
%engaging teaching practices in order to improve learning in light of increased
%Computer Science enrollment and diversity in Computer Science
%programs~\cite{vanderhydescaffolding}.  Stanier discusses further success in
%using scaffolds to introduce new topics in a higher education
%environment~\cite{stanierhighered}.
