\section{Related Work}
\label{SEC:related-work}

There is some history of using card magic to convey or explain computer science
concepts.

\paragraph{Error Correction and Parity}
Early work under the ``Computer Science Unplugged'' umbrella by Bell et al. uses
a 5 by 5 grid of cards in an exercise designed to teach about parity and
error correction~\cite{bell2009computer}~\cite{csunplugged}.
Greenberg et al. expand upon this style of trick by creating more advanced
versions with larger grids.  Some new versions use additional computer science
techniques to increase the size of the grid that can be handled by a human
magician.  Others rely on software assistance to handle more complex
computation~\cite{Greenberg2017}~\cite{Greenberg2018}.

\paragraph{Algorithm Analysis and Design}
Ferreira et al. use ``self-working'' card trick called ``Are You Psychic''
as a medium to explain topics
like problem decomposition, pre- and post- conditions, and
invariants~\cite{ferreira2014magic}.
The trick's relevance to each lesson is illustrated by
mapping each of its steps.
onto a formal description of the trick's algorithm.

\paragraph{Multi-trick Magic Shows}
Garcia et al. have produced three papers describing a wide variety of magic
tricks along the computer sciences concepts they can help
teach~\cite{garcia2012demystifying}
~\cite{garcia2013demystifying}
~\cite{garcia2016demystifying}.
Similarly, Curzon et al. found success explaining computer
science concepts to younger students using magic shows~\cite{Curzon2008}.
